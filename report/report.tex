\documentclass{article}

\usepackage[english]{babel}

\usepackage[letterpaper,top=2cm,bottom=2cm,left=3cm,right=3cm,marginparwidth=1.75cm]{geometry}

\usepackage{amsmath}
\usepackage{graphicx}
\usepackage[colorlinks=true, allcolors=blue]{hyperref}
\usepackage{natbib}
\bibliographystyle{alpha}
\usepackage{caption}
\usepackage{float}

\title{Redes de Computadores \\ \large Trabalho Prático 1}
\author{Luís Felipe Ramos Ferreira}
\date{\href{mailto:lframos\_ferreira@outlook.com}{\texttt{lframos\_ferreira@outlook.com}}
}

\begin{document}

\maketitle

\section{Introdução}

O Trabalho Prático 1 da disciplina de Redes de Computadores teve como proposta
o desenvolvimento de um \textit{Campo Minado} que permite a interação entre um
cliente e um servidor usando sockets em linguagem C.

O repositório onde está armazenado o código utilizado durante o desenvolvimento
desse projeto
pode ser encontrado \href{https://github.com/lframosferreira/tp1-redes}{neste
      endereço}.

\section{Implementação}

Conforme especificado no enunciado, o projeto foi todo desenvolvido na
linguagem de programação C em um ambiente \textit{Linux}, e o manuseio de
sockets por
meio da interface POSIX
disponibilizada para a linguagem. Para manter uma maior organização do código,
além dos arquivos \textit{server.c} e \textit{client.c}, os quais possuem
respectivamente as implementações do servidor e do cliente,
foi criado um arquivo auxiliar \textit{common.c} e seu arquivo de cabeçalho
\textit{common.h}, os quais possuem as especificações e implementações de
funções auxiliares que podem ser utilizadas por ambas as partes do projeto.

\section{Desafios, dificuldades e imprevistos}

A primeira dificuldade imposta pelo trabalho prático foi a familiarização com a
interface POSIX de programação em sockets. Embora a linguagem C seja
considerada de alto nível, em muitos momentos suas funcionalidades podem ser
confusas, o que tornou difícil um primeiro contato com a criação dos sockets e
da conexão entre eles. No entanto, esse empecilho não durou muito, uma vez que
existe uma infinidade de conteúdos sobre o assunto disponibilizados na
\textit{internet}, além, é
claro, das páginas de manual disponibilizadas nas distribuições \textit{Linux}.
Outra fonte extremamente útil para lidar com dificuldades do tipo foi a
\textit{playlist} do professr Ítalo Cunha, disponibilizada na especificação do
trabalho.

Outro desafio encontrado durante o desenvolvimento foi a de manter a
flexibilidade da escolha entre \textit{IPv4} e \textit{IPv6} na comunicação
entre o cliente e o servidor. Em particular, compreender como identificar, por
parte do cliente, qual tipo de endereço estava sendo utilizado, tomou bastante
tempo. No fim,
graças mais uma vez à \textit{playlist} do professor Ítalo o problema foi
resolvido. A função \textit{addrparse()} desenvolvida pelo professor, que
utiliza a função \textit{inet\_pton()} para converter da \textit{string} que
representa o endereço para o formato binário correto
facilitou a resolução desse problema.

A lógica de conexão e saída de diferentes clientes com o servidor também foi um
desafio no desenvolvimento do projeto. A princípio,
na primeira implementação, assumi que após um cliente enviar o comando
\textit{exit} o servidor deveria ser desligado também. No entanto, essa lógica
esta
incorreta, uma vez que o comportamento esperado é de que, após um cliente se
desconectar, o servidor se mantenha rodando para que novos clientes possam se
conectar e jogar uma partida de
campo minado. Para resolver esse problema, dois laços foram utilizados por
parte do servidor. No primeiro laço, utiliza-se a função \textit{accept()} para
aceitar a conexão com
algum cliente que esteja tentando se conectar com o servidor. Assim que uma
conexão for feita, um novo laço é feito, pelo qual a comunicação entre cliente
e servidor é feita. Agora, quando
o cliente atual envia uma mensagem de \textit{exit}, a conexão entre o servidor
e este cliente é fechada, mas o servidor continua funcionando, e executa a ação
bloqueante do \textit{accept()} enquanto aguarda por um novo cliente para
se conectar. Desse modo, o servidor só irá parar de funcionar quando acontecer
um erro durante a execução ou quando o usuário com controle de acesso sobre o
servidor forçadamente parar sua execução.

Por fim, um dos outros grandes desafios encontrados durante o desenvolvimento
do trabalho foi o do envio da estrutura de \textit{action} nas mensagens
trocadas entre
o cliente e o servidor. Em um primeiro momento, a estrutura foi passado
diretamente nas mensagens, utilizando-se a função \textit{send()} com o o
\textit{buffer} de mensagem
sendo a própria estrutura de \textit{action}. Após discussões no fórum da
disciplina, foi levantada a questão de se utilizar
serialização para envio da estrutura. Fiz uma implementação inicial dessa
serialização com base em recomendações de conteúdos
disponibilizados na \textit{internet}. Apesar de ter funcionado, pelas
respostas obtidas pela monitora no fórum, ficou-se entendido que
também seria correto não utilizar serialização, o que tornaria o código mais
simples e compatível com o dos outros colegas que não utilizaram
da mesma forma de implementação da serialização.

\section{Conclusão}

Em suma o projeto permitiu grandes aprendizados tanto na parte teórica como na
parte prática do desenvolvimento de sistemas de redes. As implementações
tornaram
possível compreender melhor como funciona o protocolo de comunicação TCP, como
deve ser feita e mantida a comunicação entre um servidor e um cliente, etc. Na
parte prática, foi permitido obter uma maior familiaridade
com a interface POSIX de programação em Sockets, assim como em programação na
linguagem C de maneira geral.

\section{Referências}

\begin{itemize}
      \item Livros:
            \begin{itemize}
                  \item Tanenbaum, A. S. \& Wetherall, D. (2011), Computer
                        Networks, Prentice Hall, Boston.
                  \item TCP/IP Sockets in C\@: Practical Guide for Programmers,
                        Second Edition
            \end{itemize}

      \item Web:
            \begin{itemize}
                  \item

                        \url{https://www.tutorialkart.com/c-programming/c-read-text-file/#gsc.tab=0}
                  \item

                        \url{https://www.gnu.org/software/libc/manual/html_node/Example-of-Getopt.html}
                  \item

                        \url{https://riptutorial.com/c/example/30858/using-gnu-getopt-tools}
                  \item

                        \url{https://www.ibm.com/docs/en/zos/2.3.0?topic=sockets-using-sendto-recvfrom-calls}
                  \item

                        \url{https://www.educative.io/answers/how-to-implement-tcp-sockets-in-c}
                  \item

                        \url{https://www.geeksforgeeks.org/regular-expressions-in-c/}
            \end{itemize}

      \item Youtube:
            \begin{itemize}
                  \item \href{https://www.youtube.com/@JacobSorber}{Jacob
                              Sorber}
                  \item

                        \href{https://www.youtube.com/watch?v=_lQ-3S4fJ0U&list=PLPyaR5G9aNDvs6TtdpLcVO43_jvxp4emI}{Think
                              and Learn sockets playlist}
                  \item

                        \href{https://www.youtube.com/watch?v=tJ3qNtv0HVs&t=2s}{Playlist do professor
                              Ítalo Cunha}
            \end{itemize}

\end{itemize}

\end{document}
