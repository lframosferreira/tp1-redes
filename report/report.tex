\documentclass{article}

\usepackage[english]{babel}

\usepackage[letterpaper,top=2cm,bottom=2cm,left=3cm,right=3cm,marginparwidth=1.75cm]{geometry}

\usepackage{amsmath}
\usepackage{graphicx}
\usepackage[colorlinks=true, allcolors=blue]{hyperref}
\usepackage{natbib}
\bibliographystyle{alpha}
\usepackage{caption}
\usepackage{float}

\title{Redes de Computadores \\ Trabalho Prático 1}
\author{Luís Felipe Ramos Ferreira \\ 2019022553 \\
    \href{mailto:lframos_ferreira@outlook.com}{\texttt{lframos\_ferreira@outlook.com}}}
\begin{document}
\maketitle

\section{Introdução}

O Trabalho Prático 1 da disciplina de Redes de Computadores

\begin{figure}[H]
    \centering
    \includegraphics[width=0.6\textwidth]{images/gengar.png}
    \caption{Teste de imagem}
\end{figure}

A primeira coisa

\section{Conclusão}

Em suma

\section{Referências}

\begin{itemize}
    \item Livros:
          \begin{itemize}
              \item tanembaum
              \item sockets
          \end{itemize}

          Web:
          https://www.tutorialkart.com/c-programming/c-read-text-file/#gsc.tab=0
          https://www.gnu.org/software/libc/manual/html_node/Example-of-Getopt.html
          https://riptutorial.com/c/example/30858/using-gnu-getopt-tools

          Youtube:
          jacob sorber

          https://www.youtube.com/watch?v=_lQ-3S4fJ0U&list=PLPyaR5G9aNDvs6TtdpLcVO43_jvxp4emI
\end{itemize}

\end{document}