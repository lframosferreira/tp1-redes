\documentclass{article}

\usepackage[english]{babel}

\usepackage[letterpaper,top=2cm,bottom=2cm,left=3cm,right=3cm,marginparwidth=1.75cm]{geometry}

\usepackage{amsmath}
\usepackage{graphicx}
\usepackage[colorlinks=true, allcolors=blue]{hyperref}
\usepackage{natbib}
\bibliographystyle{alpha}
\usepackage{caption}
\usepackage{float}

\title{Redes de Computadores \\ \large Trabalho Prático 1}
\author{Luís Felipe Ramos Ferreira}
\date{\href{mailto:lframos\_ferreira@outlook.com}{\texttt{lframos\_ferreira@outlook.com}}
}

\begin{document}

\maketitle

\section{Introdução}

O Trabalho Prático 1 da disciplina de Redes de Computadores teve como proposta
o desenvolvimento de um \textit{Campo Minado} que permite a interação entre um
cliente e um servidor usando sockets em linguagem C.

\begin{figure}[H]
      \centering
      \includegraphics[width=0.6\textwidth]{images/gengar.png}
      \caption{Teste de imagem}
\end{figure}

\section{Implementação}

Conforme especificado no enunciado, o projeto foi todo desenvolvido na
linguagem de programação C em um ambiente Linux, e o manuseio de Sockets por
meio da interface POSIX
disponibilizada para a linguagem. Para manter uma maior organização do código,
além dos arquivos \textit{server.c} e \textit{client.c}, os quais possuem
respectivamente, as implementações do servidor e do cliente,
foi criado um arquivo auxiliar \textit{common.c} e seu arquivo de cabeçalho
\textit{common.h}, os quais possuem as especificações e implementações de
funções auxiliares que podem ser utilizadas por ambas as partes do projeto.

\section{Desafios, dificuldades e imprevistos}

A primeira dificuldade imposta pelo trabalho prático foi a familiarização com a
interface POSIX de programação em Sockets. Embora a linguagem C seja
considerada de alto nível, em muitos momentos suas funcionalidades podem ser
confusas, o que tornou difícil um primeiro contato com a criação dos sockets e
da conexao entre eles. No entanto, esse empecilho não durou muito, uma vez que
existe uma infinidade de conteúdos sobre o assunto disponibilizados na
\textit{internet}, além, é
claro, das páginas de manual disponibilizadas nas distribuições \textit{Linux}.
Outra fonte extremamente útil para lidar com dificuldades do tipo foi a
\textit{playlist} do professr Ítalo Cunha, disponibilizada na especificação do
trabalho.

Outro desafio encontrado durante o desenvolvimento foi 

- parse de ipv4 e ipv6

- como lidar com clientes se desconectando e conectando novamente



\section{Conclusão}

Em suma o projeto permitiu grandes aprendizados tanto na parte teórica como na
parte prática do desenvolvimento de sistemas de redes. As implementações
tornaram
possível compreender melhor como funciona o protocolo de comunicação TCP, como
deve ser feita e mantida a comunicação entre um servidor e um cliente, etc. Na
parte prática, foi permitido obter uma maior familiaridade
com a interface POSIX de programação em Sockets, assim como em programação na
linguagem C de maneira geral.

\section{Referências}

\begin{itemize}
      \item Livros:
            \begin{itemize}
                  \item Tanenbaum, A. S. \& Wetherall, D. (2011), Computer
                        Networks, Prentice Hall, Boston.
                  \item TCP/IP Sockets in C\@: Practical Guide for Programmers,
                        Second Edition
            \end{itemize}

      \item Web:
            \begin{itemize}
                  \item

                        \url{https://www.tutorialkart.com/c-programming/c-read-text-file/#gsc.tab=0}
                  \item

                        \url{https://www.gnu.org/software/libc/manual/html_node/Example-of-Getopt.html}
                  \item

                        \url{https://riptutorial.com/c/example/30858/using-gnu-getopt-tools}
                  \item

                        \url{https://www.ibm.com/docs/en/zos/2.3.0?topic=sockets-using-sendto-recvfrom-calls}
                  \item

                        \url{https://www.educative.io/answers/how-to-implement-tcp-sockets-in-c}
                  \item

                        \url{https://www.geeksforgeeks.org/regular-expressions-in-c/}
            \end{itemize}

      \item Youtube:
            \begin{itemize}
                  \item \href{https://www.youtube.com/@JacobSorber}{Jacob
                              Sorber}
                  \item

                        \href{https://www.youtube.com/watch?v=_lQ-3S4fJ0U&list=PLPyaR5G9aNDvs6TtdpLcVO43_jvxp4emI}{Think
                              and Learn sockets playlist}
                  \item

                        \href{https://www.youtube.com/watch?v=tJ3qNtv0HVs&t=2s}{Playlist do professor
                              Ítalo Cunha}
            \end{itemize}

\end{itemize}

\end{document}